\documentclass[a4paper, 12pt]{article}

\usepackage{amsmath}

\title{Mathematics III Assignment 1}
\author{Zubair Abid, 20171076}

\begin{document}
\maketitle

\begin{enumerate}

	\item 
	$X$ is a random variable with $PDF$ given by
	
	\[ f(n) = 
		\begin{cases}
			cx^2 & \quad \text{if } x \leq 1 \\
			0 & \quad \text{otherwise}
		\end{cases}
	\]

	\begin{enumerate}
		\item Constant $c$
		\\ \\
		Summation of PDF over domain adds up to 1.\\
		Or, 
		\[
		\int\limits_{-\infty}^{\infty} cx^2 = 1
		\]
		Since the function returns $0$ everywhere except
		 at $[-1, 1]$, 	we just calculate
		\begin{align*}
			\int\limits_{-1}^{1} cx^2 &= 1\\
			c\times \left.\frac{x^3}{3}\right|_{-1}^{1} &= 1\\
			c\times \frac{2}{3} &= 1\\
			c &= 1.5 \text{ (Answer)}
		\end{align*}				
		
		\item $E\left[X\right]$ and $Var\left(X\right)$
		$E[X]$ is 
		\begin{align*}
			\int\limits_{-1}^{1} xcx^2
			&= 1.5 \times \int\limits_{-1}^{1} x^3\\
			&= 1.5 t\times \left.\frac{x^4}{4}\right|_{-1}^{1}\\
			&= 0
		\end{align*}
			
		Now, $Var(X) = E[x^2] - (E[X])^2$, or\\
		\begin{align*}
			Var(X) &= 1.5\times \int\limits_{-1}^{1} x^4 - 0\\
			&= 1.5\times \left.\frac{x^5}{5}\right|_{-1}^{1} \\
			&= 1.5 \times 0.4\\
			&= 0.6
		\end{align*}
		
		
		\item $P\left(X \geq \frac{1}{2} \right)$
		\\ \\ 
		Since the function given is a PDF, to get the
		$P(X \geq \frac{1}{2})$, all we need to do is
		integrate $f(x)$ from $\frac{1}{2}$ to $1$\\
		Or, 
		\begin{align*}
			P\left(X \geq \frac{1}{2} \right) &=
			\int\limits_{\frac{1}{2}}^{1} cx^2\\
			&= 1.5 \times \left.\frac{x^3}{3}
			\right|_{\frac{1}{2}}^{1}\\
			&= 1.5 \times \frac{7}{24}\\
			&= 0.4375
		\end{align*}
	\end{enumerate}

	\item Given, the CDF is:
	\[
		F(x) = \frac{x^3 + k}{40} \quad x = 1, 2, 3
	\]
	\begin{enumerate}
		\item Value of k\\
		Since $F(x)$ is a CDF, value of $F(3) = 1$\\
		or 
		\begin{align*}
			\frac{27+k}{40} = 1\\
			k = 13 \quad\text{(Q.E.D)}
		\end{align*}

		\item Find the probability distribution of X\\
		This can be obtained by simple subtraction,
		answer is
		\begin{align*}
			P(X=1) &= \frac{1+13}{40}\\
			&= \frac{7}{20}\\
			P(X=2) &= \frac{21-14}{40}\\
			&= \frac{7}{40}\\
			P(X=3) &= \frac{40-21}{40}\\
			&= \frac{19}{40}
		\end{align*}
		
	\end{enumerate}
	

\end{enumerate}

\end{document}