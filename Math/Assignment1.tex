\documentclass[a4paper, 12pt]{article}

\usepackage{amsmath}

\title{Mathematics III Assignment 1}
\author{Zubair Abid, 20171076}

\begin{document}
\maketitle

\begin{enumerate}

	% Problem 1
	\item 
	$X$ is a random variable with $PDF$ given by
	
	\[ f(n) = 
		\begin{cases}
			cx^2 & \quad \text{if } x \leq 1 \\
			0 & \quad \text{otherwise}
		\end{cases}
	\]

	\begin{enumerate}
		\item Constant $c$
		\\ \\
		Summation of PDF over domain adds up to 1.\\
		Or, 
		\[
		\int\limits_{-\infty}^{\infty} \mathrm{cx^2}
		\, \mathrm{d}x = 1
		\]
		Since the function returns $0$ everywhere except
		 at $[-1, 1]$, 	we just calculate
		\begin{align*}
			\int\limits_{-1}^{1} \mathrm{cx^2}\, 
			\mathrm{d}x &= 1\\
			c\times \left.\frac{x^3}{3}\right|_{-1}^{1} &= 1\\
			c\times \frac{2}{3} &= 1\\
			c &= 1.5 \text{ (Answer)}
		\end{align*}				
		
		\item $E\left[X\right]$ and $Var\left(X\right)$
		$E[X]$ is 
		\begin{align*}
			\int\limits_{-1}^{1} \mathrm{xcx^2}\,\mathrm{d}x
			&= 1.5 \times \int\limits_{-1}^{1} 
			\mathrm{x^3}\,\mathrm{d}x\\
			&= 1.5 t\times \left.\frac{x^4}{4}\right|_{-1}^{1}\\
			&= 0
		\end{align*}
			
		Now, $Var(X) = E[x^2] - (E[X])^2$, or\\
		\begin{align*}
			Var(X) &= 1.5\times \int\limits_{-1}^{1} 
			\mathrm{x^4}\,\mathrm{d}x - 0\\
			&= 1.5\times \left.\frac{x^5}{5}\right|_{-1}^{1} \\
			&= 1.5 \times 0.4\\
			&= 0.6
		\end{align*}
		
		
		\item $P\left(X \geq \frac{1}{2} \right)$
		\\ \\ 
		Since the function given is a PDF, to get the
		$P(X \geq \frac{1}{2})$, all we need to do is
		integrate $f(x)$ from $\frac{1}{2}$ to $1$\\
		Or, 
		\begin{align*}
			P\left(X \geq \frac{1}{2} \right) &=
			\int\limits_{\frac{1}{2}}^{1} 
			\mathrm{cx^2}\,\mathrm{d}x\\
			&= 1.5 \times \left.\frac{x^3}{3}
			\right|_{\frac{1}{2}}^{1}\\
			&= 1.5 \times \frac{7}{24}\\
			&= 0.4375
		\end{align*}
	\end{enumerate}

	% Problem 2
	\item Given, the CDF is:
	\[
		F(x) = \frac{x^3 + k}{40} \quad x = 1, 2, 3
	\]
	\begin{enumerate}
		\item Value of k\\
		Since $F(x)$ is a CDF, value of $F(3) = 1$\\
		or 
		\begin{align*}
			\frac{27+k}{40} = 1\\
			k = 13 \quad\text{(Q.E.D)}
		\end{align*}

		\item Find the probability distribution of X\\
		This can be obtained by simple subtraction,
		answer is
		\begin{align*}
			P(X=1) &= \frac{1+13}{40}\\
			&= \frac{7}{20}\\
			P(X=2) &= \frac{21-14}{40}\\
			&= \frac{7}{40}\\
			P(X=3) &= \frac{40-21}{40}\\
			&= \frac{19}{40}
		\end{align*}
		
		\item Given $Var(X) = \frac{259}{320}$,
		 calculate $Var(4X-5)$		
		 \\
		 \\
		 \begin{align*}
		 	{\sigma_{ax+b}}^2 &= a^2 \times {\sigma_x}^2\\
		 	{\sigma_{4x+5}}^2 &= 16 \times {\frac{259}{320}}\\
		 	&= \frac{259}{20}\\
		 	&= 12.95
		 \end{align*}
	\end{enumerate}
	
	% Problem 3
	\item 
	\begin{align*}
		P \left(\text{First 6 in 2nd throw }\right| 
		\left.\text{First 6 on even throw}\right)
		&= \frac{P\left(\text{2nd throw}\right)\cap
		P\left(\text{even throw}\right)}{P\left(
		\text{even throw}\right)}\\
		&= \frac{\frac{5}{6}\times\frac{1}{6}}
		{\frac{1}{6}\times\left[\sum \frac{5}{6}
		+ \left(\frac{5}{6}\right)^3 + \cdots\right]}\\
		&= \frac{\frac{5}{36}}{\frac{1}{6}\times
		\frac{30}{11}}\\
		&= \frac{\frac{5}{6}}{\frac{30}{11}}\\
		&= \frac{5}{6} \times \frac{11}{30}\\
		&= \frac{11}{36} \left(\text{Answer}\right)
	\end{align*}
	
	% Number 4
	\item By Bayes' Theorerm,\\
	\begin{align*}
		P( 0\text{ sent}|0 \text{ received})
		&= \frac{P(0\text{ received}|0 \text{ sent}
		)\times P(0\text{ sent}) }
		{P(0\text{ received})}\\		
		&= \frac{\frac{4}{5}\times\frac{2}{3}}
		{P(0\text{ received}|0\text{ sent})\times
		P(0\text{ sent}) + P(0\text{ received}|
		1\text{ sent})\times P(1\text{ sent})}\\
		&= \frac{\frac{8}{15}}{\frac{8}{15}\times
		\frac{1}{18}}\\
		&= \frac{\frac{8}{15}}{\frac{53}{90}}\\
		&= \frac{48}{53} \quad\text{(Ans)}
	\end{align*}
	
	% Number 5
	\item 
	
	\begin{enumerate}
		\item
		Sum of a PDF over its given range is 1\\
		Given function:
		\[ f(n) = 
			\begin{cases}
				cx^2 & \quad \text{if } 0 < x < 3 \\
				0 & \quad \text{otherwise}
			\end{cases}
		\]
	
		\begin{align*}
			\int_{0}^{3} \mathrm{cx^2}\,\mathrm{d}x &= 1\\
			c\times\left.\frac{x^3}{3}\right|_{0}^{3} &= 1\\
			c\times 9 &= 1\\
			c &= \frac{1}{9}
		\end{align*}
	
		\item $P(1<X<2)$
		\begin{align*}
			P(1<X<2) &= \frac{1}{9} \times 
			\int_{1}^{2} \mathrm{x^2}\,\mathrm{d}x\\
			&= \frac{1}{9} \times \left.\frac{x^3}{3}
			\right|_{1}^{2}\\
			&= \frac{1}{9} \times \left[ \frac{8}{3}
			- \frac{1}{3}\right]\\
			&= \frac{7}{27}
		\end{align*}
	\end{enumerate}		
	
	% Problem 6
	\item 
		\begin{enumerate}
		\item
			\begin{align*}
			c\times\int_{-\infty}^{\infty} \mathrm{
			\frac{1}{x^2+1}}\,\mathrm{d}x &= 1\\
			c\times\left.tan^{-1} (x)\right|_{-\infty}^
			{\infty} &= 1\\
			c \times \Pi &= 1\\
			c &= \frac{1}{\Pi}
			\end{align*}
		\item 
			\begin{align*}
			x \in \left(-1, -\sqrt{\frac{1}{3}}\right)
			\cup \left(\sqrt{\frac{1}{3}}, 1\right)\\
			P\left(\frac{1}{3}<x^2<1\right) &= \frac{1}{\Pi}
			\times\left[\int_{-1}^{-\sqrt{\frac{1}{3}}} 
			\mathrm{\frac{1}{1+x^2}}\,\mathrm{d}x + 
			\int_{\sqrt{\frac{1}{3}}}^{1} 
			\mathrm{\frac{1}{1+x^2}}\,\mathrm{d}x\right]\\
			&= \frac{1}{\Pi} \times \left[
			\left.tan^{-1} (x)\right|_{-1}^{-\sqrt{\frac{1}{3}}}
			+ \left.tan^{-1} (x)\right|_{\sqrt{\frac{1}{3}}}^{1}
			\right]\\
			&= \frac{1}{\Pi} \times \left[ -\frac{\Pi}{6}
			+ \frac{\Pi}{4} + \frac{\Pi}{4} - \frac{\Pi}{6} 
			\right]\\
			&= \frac{1}{\Pi} \times \frac{\Pi}{6}\\
			&= \frac{1}{6}
			\end{align*}
		\end{enumerate}

	% Problem 7
	\item 
	\begin{enumerate}
		\item For values of $x > 0$
		\begin{align*}
			\int_{0}^{x} \mathrm{f(x)}\,\mathrm{d}x &=
			F(x),\quad F(x) = 1 - e^{-2x}\\
			f(x) &= \frac{dF(x)}{dx}\\
			f(x) &= 2e^{-2x}
		\end{align*}
			
		for values of $x < 0$, $f(x) = 0$\\
		Answer:
		\[
			\text{PDF }f(x) = 
			\begin{cases}
				2e^{-2x} \quad \text{if x} > 0\\
				0 \quad \text{otherwise}
			\end{cases}
		\]
			
		\item For $P(X>2)$, we do $F(\infty)-F(2)$
		\begin{align*}
			F(\infty) - F(2) &= 1 - \left(1-e^{-4}\right)\\
			&= e^{-4} \quad\text{(Ans)}
		\end{align*}
			
		\item $P(-3<X\leq4) = P(X\leq4)$
		\begin{align*}
		&= F(4)\\
		&= 1 - e^{-8} \quad\text{(Ans)}
		\end{align*}
		
	\end{enumerate}

	% Problem 8
	\item 
		\begin{enumerate}
			\item 
			\begin{align*}
				F1(-\infty) = 0\\
				F1(\infty) = 1\\
				\frac{d}{dx} F1(x) \geq 0 \forall x \subset domain\\
				\Rightarrow \textrm{the function is a CDF}
			\end{align*}
			\item 
			\begin{align*}
				F1(-\infty) \neq 0\\
				F1(\infty) = 1\\
				\Rightarrow \textrm{the function is not a CDF}
			\end{align*}
		\end{enumerate}

	% Problem 9
	\item B = black, O = orange, W = white
	\\
	\textbf{Assumption}: the balls are picked out one by 
	one, so the following combinations are all possible and distinct\\
	possible outcomes = BB(+4), WW(-2), OO(0), BW(1), WB(1), BO(2), 
	OB(2), WO(-1), OW(-1)\\
	\\
	Outcome is +4: 
	\begin{align*}
		\textrm{BB} &= \frac{4}{14} \times \frac{3}{13}\\
		&= \frac{6}{91}
	\end{align*}
	\\
	Outcome is -2: 
	\begin{align*}
		\textrm{WW} &= \frac{8}{14} \times \frac{7}{13}\\
		&= \frac{28}{91}
	\end{align*}
	\\
	Outcome is 0: 
	\begin{align*}
		\textrm{OO} &= \frac{2}{14} \times \frac{1}{13}\\
		&= \frac{1}{91}
	\end{align*}
	\\
	Outcome is +1: 
	\begin{align*}
		\textrm{BW and WB} &= 2\times\frac{4}{14} \times \frac{8}{13}\\
		&= \frac{16}{91}\\
		&= \frac{32}{91}\\
	\end{align*}
	\\
	Outcome is +2: 
	\begin{align*}
		\textrm{BO and OB} &= 2\times\frac{4}{14} \times \frac{2}{13}\\
		&= \frac{4}{91}\\
		&= \frac{8}{91}\\
	\end{align*}
	\\
	Outcome is -1: 
	\begin{align*}
		\textrm{WO and OW} &= 2\times\frac{8}{14} \times \frac{2}{13}\\
		&= \frac{8}{91}\\
		&= \frac{16}{91}\\
	\end{align*}
	
	% Problem 10
	\item 
	\begin{align*}
		P(X=0) &= \frac{\binom{5}{2}\times 3!}{5!}\\
		&= \frac{1}{2}\\
		\textrm{similarly, }\\
		P(X=1) &= \frac{\binom{5}{3}\times 2!}{5!}\\
		&= \frac{1}{6}\\
		P(X=2) &= \frac{\binom{5}{4}\times 2! \times 1!}{5!}\\
		&= \frac{1}{12}\\
		P(X=3) &= \frac{\binom{5}{5}\times 3!}{5!}\\
		&= \frac{1}{20}\\
		P(X=4) &= \frac{4!}{5!}\\
		&= \frac{1}{5}\\
	\end{align*}
	
	% Problem 11
	\item Assume X is the number of defective items
	in the sample\\
	or, possible values of $X = 0, 1, 2, 3$\\
	
	\begin{align*}
		E(X) &= \sum xf(x)\\
		&= f(1) + 2\times f(2) + 3 \times f(3)\\
		&= \frac{4\times16\times15}{20\times19\times18}
		+ 2\times\frac{4\times3\times16}{20\times19\times18}
		+ 3\times\frac{4\times3\times2}{20\times19\times18}\\
		&= \frac{1416}{6840}\\
		&= 0.207 \quad\text{(Ans)}
	\end{align*}
	
	% Problem 12
	\item For errors, at least three bits should be flipped\\
	Therefore, probability \\
	\begin{align*}
	&= flip(3) + flip(4) + flip(5)\\
	&= \binom{5}{3}\times 0.8^2 \times 0.2^3 +
	\binom{5}{4} \times 0.8 \times0.2^4 +
	\binom{5}{5} \times 0.2^5\\
	&= 0.0512 + 0.0064 + 0.00032\\
	&= 0.05792 \quad \text{ (Ans)}
	\end{align*}
	
	% Problem 13
	\item 
	\begin{align*}
		P(X=1) &= \frac{\binom{5}{1}\times9!}{10!}\\
		&= \frac{1}{2}\\ \\
		P(X=2) &= \frac{\binom{5}{1}\times\binom{5}{1}\times8!}{10!}\\
		&= \frac{5}{18}\\ \\
		P(X=3) &= \frac{(2!\times\binom{5}{2})\times\binom{5}{1}\times7!}{10!}\\
		&= \frac{5}{36}\\ \\
		P(X=4) &= \frac{(3!\times\binom{5}{3})\times\binom{5}{1}\times6!}{10!}\\
		&= \frac{5}{84}\\ \\
		P(X=5) &= \frac{(4!\times\binom{5}{4})\times\binom{5}{1}\times5!}{10!}\\
		&= \frac{5}{252}\\ \\
		P(X=6) &= \frac{(5!\times\binom{5}{5})\times\binom{5}{1}\times4!}{10!}\\
		&= \frac{1}{252}\\ \\
		P(X >= 7) &= 0
	\end{align*}
	
	% Problem 14
	\item 
	\begin{align*}
		\frac{\textrm{Number of ways to have 3 men in each team}}
		{\textrm{All possible combinations}}\\
		&= \frac{\binom{6}{3}\times\binom{6}{3}}
		{\binom{12}{6}}\\
		&= \frac{400}{924}\\
		&= 0.4329
	\end{align*}
	
	% Problem 15
	\item 
	
	% Problem 16
	\item 
	\begin{align*}
		Cov(X,Y) &= E(XY)-E(X)-E(Y)\\
		\\
		E(X) &= \int_0^y \int_0^1 \textrm{2x} \textrm{dx dy}\\
		&= \int_0^y \left. x^2 \right|_0^{1-y} \textrm{dy}\\
		&=  
	\end{align*}
	
	% Problem 17
	\item 
	
\end{enumerate}

\end{document}